\documentclass{article}
\usepackage{amsmath}
\title{Reframing Hegselmann-Krause Dynamics with conformity}
\author{Brandon Zerbe}
\begin{document}
\maketitle
\section{Differential reformulation}

In this section I reformulate Hegselmann-Krause (HK) dynamics in terms of differentials,
a concept from calculus that is widely used in physics and engineering.
The formulation of HK dynamics you use is known as difference equations, and I argue that these difference
equations really arise from Euler's method of integration where the difference can be
expressed as a differential.  This reformulation is important because it provides standard
tools that allow further analysis of the model without resorting to simulation.

I also introduce slightly modified notation.  Specifically, I write values in position space,
what you denote as $o_i$, as vectors, ${\vec{o}}_i$, which can be thought of as an arrow from $0$
to $o_i$.  This is standard notation although it is often not done for 1D systems; 
I choose to use the vector notation here 
for three reasons: (1) it allows us to keep track of the sign of
differences, (2) it will allow us to use the inner product to simplify some of the later analysis we do,
and (3) it makes generalization to higher dimensions easier.  Note, with this notation,
$o_i = |{\vec{o}}_i|$, so parameters that are otherwise vectors but which do not have vector symbols
denote the magnitude of a distance.

HK dynamics difference are defined in your 2023 paper, which I summarize as
\begin{subequations}
    \begin{align}
        {\vec{d}}_{i,j}(t) = {\vec{o}}_j(t) - {\vec{o}}_i(t) \label{eq:diff-vec}\\
        {\vec{o}}_i(t) &= {\vec{o}}_i(t-1) + \alpha_i' \frac{1}{|N_i(t-1)|} \sum_{j\in N_i(t-1)}  {\vec{d}}_{i,j}\label{eq:opin-diff}\\
        N_i(t) &= {j: d_{i,j}(t) < \epsilon_i} \label{eq:neighborhood}
    \end{align}
\end{subequations}    
Note that only Eq. \ref{eq:opin-diff} relates the values of parameters at time $t$ to 
to values of parameters at time $t - 1$; that is, the left hand side of this expression is the value of the
opinion vector parameter at time $t$ while the right hand side is an expression evaluated at time $t-1$.
Eq. \ref{eq:opin-diff} is what I mean when I say difference equation.  The other two expressions define the
"difference vector" (Eq. \ref{eq:diff-vec}) and the "neighborhood of $i$" (Eq. \ref{eq:neighborhood}), and note that
both sides of these expression are evaluated at the same "time snapshot".  
Henceforth, I will only explicitly
use the time dependence notation when writing the difference equation, and for brevity
I will make the time dependence in expression like Eqs. \ref{eq:diff-vec} and \ref{eq:neighborhood}
implicit.  That is ${\vec{d}}_{i,j} = {\vec{o}}_j - {\vec{o}}_i$ and $N_i = {j: d_{i,j} < \epsilon_i}$
are the respective versions of these two equations with implicit time dependence.
This is standard notational choices in analysis, but if it leads to confusion about 
what depends on time, don't hesitate to ask.

Note that Eq. \ref{eq:opin-diff} can be re-written somewhat by analyzing the sum.
Namely, $ \frac{1}{|N_i|} \sum_{j\in N_i} {\vec{o}}_j$ is the mean of all of the opinions
within $\epsilon_i$ of agent $i$.  Denote this mean as
\begin{align}
    {\vec{{\bar{o}}}}_i = \frac{1}{|N_i|} \sum_{j\in N_i} {\vec{o}}_j\label{eq:neighborhood-mean}
\end{align}
which again implicitly depends on time.  Call this the mean opinion in $i$'s neighborhood.

Further,  
my reading of Eq. \ref{eq:opin-diff} is that it assumes that the
timestep is sufficiently small to apply Euler integration as a strategy to obtain a
dynamic solution for the opinions.  If you are unfamiliar,
Euler integration takes the form of 
\begin{align}
    x(t+dt) &\approx x(t) + \frac{dx}{dt}(t) dt\label{eq:euler-integration}
\end{align}
which converges to the true answer (up to compuational truncation error) as $dt$, the time step,
is taken smaller and smaller.  Here $\frac{dx}{dt}$ is known as a "differential", which
was a concept introduced about 400 years ago independently by Newton and Leibniz; the analysis of
differentials is commonly called "calculus" [which I personally see as a hubrus as calculus should
really refer to any type of calculation, but that's how we use the word now]
and when $x$ is a vector, the analysis is called "vector calculus".  Vector calculus forms the
mathematical backbone of classical physics, so vector calculus is very well understood by many people.

The difference equation is now ready to be reformulated as a differential.
By rescaling the parameter $\alpha'$ in Eq. \ref{eq:opin-diff} and comparing this equation to Eq.
\ref{eq:euler-integration}, we can write an expression for the differential
\begin{align}
    \frac{d {\vec{o}}_i}{dt} = \alpha_i ({\vec{\bar{o}}}_i - {\vec{o}}_i) \label{eq:opin-deriv}
\end{align}
Note that this expression again has implicit time dependence and the differntial can be
thought of a rate of change in time for the $i^{th}$ opinion.
That is, the agent $i$ moves toward the mean of the opions of its neighborhood at a rate proportional
to its distance from this mean.
For brevity, denote this differential by ${\vec{r}}_i = \frac{d {\vec{o}}_i}{dt}$;
physically, the interpretation of ${\vec{r}}_i$ is that the agent $i$ moves toward
the mean of the opinions within its neighborhood at a rate proportional
to its distance from the it's neighborhood mean.
Also note that I subtletly change $\alpha'$ to $\alpha$; this is intention, but
this will not change how we implement the constant parameter $\alpha_i$ in practice and
can largely be ignored unless you want to talk about finer points of calculus. 
While Eq. (\ref{eq:opin-deriv}) may look as if it could be solve via separation by parts (if
you are familiar with that technique), note that
the parameter ${\vec{\bar{o}}}_i$ is really a function of all of the opinions; as a result,
if we are really lucky and clever, we may solve this with further analysis, but in many cases
where luck and/or cleverness fails us, we will obtain solutions with $N$-particle simulations.

\subsection{TLDR}
In summary, the differential form of HK dynamics can be written as
\begin{subequations}\label{eq:update-eqs}
    \begin{align}
        {\vec{r}}_{i} &= \alpha_i ({\vec{\bar{o}}}_i - {\vec{o}}_i)\label{eq:update:rate}\\
        {\vec{o}}_i(t) &= {\vec{o}}_i(t-1) + {\vec{r}}_i(t) ~ dt\label{eq:update:opin}
    \end{align}
\end{subequations} 
These equations provide a set of ordinary differential equations that can be used to conduct Euler integration
for our $N$-particle simulation.  That is, we recursively follow the cycle of steps:
\begin{enumerate}
  {\item start off with some initial values ($o_i$) for $i \in {1, 2, ..., N-1 , N}$}
  {\item calculate the rate of change using these values using Eq. \ref{eq:update:rate}
the definition of the ${\vec{\bar{o}}}_i$ in Eq. \ref{eq:neighborhood-mean}}
  {\item update the opinion values at the next time step with Eq. \ref{eq:update:opin}}
  {\item return to step 1}
\end{enumerate}
until we are "done".  This 4-step iterative protocol is what I mean by $N$-particle
simulation, and the history of every $o_i$ in time, which is what I mean by the solution to your problem.
I assume you did $N$-particle simulations with the difference equations (without the explicit $dt$) in your previous
work,
and you may be wondering why I just spent 2 pages re-framing what you did resulting in something that looks
almost identical but with funny words sprinkled throughout.
The reason I did this reframing is that it allows us to obtain additional insight, perhaps even an analytic
approximation to the solution, which I will examine in the next section.




\section{The distribution evolution approach}
When we desire to examine stochastic effects, we will need to implement the $N$-particle simulation;
however, when we are interested in only the "average" evolution, we can adopt a full distribution approach.
This section develops the equations necessary for the full distribution approach limitted to a single $\alpha$
and single $\epsilon$; extension of this treatment to additional $\alpha$'s and $\epsilon$'s will be
done in a later section.

Instead of starting with a sample of $N$ particles drawn from some analytic distribution,
we retain the analytic distribution at time $t$ and opinion $o$
which we denote as $\rho(o, t)$.  For instance, the initial uniform- and truncated
normal- distributions
have the following function forms on $o \in [-1, 1]$, respectively:
\begin{subequations}
    \begin{align}
        \rho_{unif}(o, 0) &= \frac{1}{2}\\
        \rho_{0, norm}(o, 0) &= A e^{\frac{(o-\mu_o)^2}{2 \sigma_o^2}}
    \end{align}
\end{subequations}
where $\mu_o$ and $\sigma_o$ are standard parameters (usually the 
mean and standard deviation although
they can't be interpretted as such here) for the normal distribution
and $A$ is a normalization constant so that $\int_{-1}^1 \rho_{norm}(0) = 1$.  Again, though,
this analysis will derive the expressions needed to obtain the functional form of $\rho(o, t)$mean of th
for any $\rho(o, 0)$.
Note that outside of $[-1, 1]$, we will set $\rho_(o,0) = 0$ by the confines of our problem; this is a convenience
that allows us to simplify argumentation.

For a given $\rho(o, 0)$, we write the initial mean of the $i^{th}$ neighborhood:
\begin{align}
    {\vec{bar{o}}}(o, 0) = \int_{o - \epsilon}^{o + \epsilon} p \rho(p, 0) dp
\end{align}
which is just the definition of mean defined on the neighborhood at $o$. 
Note that this initial mean of the $i^{th}$ neighborhood
is directly calculated from $\rho(o, 0)$ and therfore can be treated as an input parameter.
The next question is "How does ${\vec{bar{o}}}(o, t)$ evolve in time?"

To answer this question, we use calculus.

This approach can be though of as weighting
every starting location on $[-1, 1]$ by $\rho(0)$; conceptually, this is equivalent to
sampling $N \to \infty$, which is why we see the stochast fluctuations, which scale as $\frac{1}{\sqrt{N}}$, fall
to zero. 
\end{document}